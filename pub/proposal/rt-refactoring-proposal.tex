\documentclass[preprint,10pt]{sigplanconf}

\newcommand{\todo}[1]{{\bfseries [[#1]]}}
%% To disable, just uncomment this line
%% \renewcommand{\todo}[1]{\relax}

\begin{document}
%
% --- Author Metadata here ---
\conferenceinfo{CSE503}{'11 Seattle, USA}
%\CopyrightYear{2007} % Allows default copyright year (20XX) to be over-ridden - IF NEED BE.
%\crdata{0-12345-67-8/90/01}  % Allows default copyright data (0-89791-88-6/97/05) to be over-ridden - IF NEED BE.
% --- End of Author Metadata ---

\title{Real-time Code Clone Refactoring Recommendations}
% 1st. author
\authorinfo{Travis Mandel}
           {University of Washington}
           {tmandel@cs.washington.edu}
% 2nd. author
\authorinfo{Todd W. Schiller}
           {University of Washington}
           {tws@cs.washington.edu}
\maketitle
\begin{abstract}
An Eclipse plugin for providing real-time refactoring suggestions
\end{abstract}

\category{D.2.6}{Software Engineering}{Programming Environments}

\keywords{refactoring, recommender system, code clones}

\section{Introduction}

Code clones are typically viewed as a problem of software
\emph{maintenance}, as failure to revise a clone can be an error. In
addition to helping developers maintain clones, this work aims to
identify (1) potential method calls and opportunities for method
extraction, eliminating the introduction of unnecessary clones, and
(2) opportunities for copy-paste coding. Both reduce development cost,
while the first also reduces maintenance cost.

\section{Evaluation}

We will evaluate the tool via user case study. Each user will be given
a series of development and maintenance tasks for an existing Java
code base. The control group will use the stock Eclipse development
environment, the tool group will use the Eclipse development
environment with the tool. For the initial study for CSE 503, each
group will consist of two individuals who have experience using the
Java development environment.

Users will be instructed that they ``own'' the code base, that is they
have permission to introduce new methods, but must document the
methods.

\subsection{Quantitative Evaluation}
For the development tasks, we will measure the time to complete the
task, the number of methods extracted, and the number of code clones
introduced into the code base, and percentage of the new code that is
a clone of other code. For the control group, we will additionally
record the number of times the developer searches for a method that
performs a subtask. For the tool group, we will record the number of
times the tool detects a code clone, the user ignores the tool's
alert, the user views the tool's analysis, and the user acts on the
tool's analysis. Additionally, when the user acts on the tool's
analysis, we will record the ranking of the clone the user acted on.

For the maintenance tasks, we will measure the time to complete the
task, and whether the user has correctly revised all of the clones (as
determined by hidden test suite). We will record the same
group-specific metrics as for the development tasks.

\subsection{Qualitative Evaluation}
In addition to the quantitative results, we will have the study
participants respond to the following questions:

\begin{itemize}
  \item If you extracted a method, how did you decide to extract the
    method? 
  \item If you chose not to extract a method, why did you decide not
    to extract the method?
\end{itemize}

\noindent Additionally, the treatment group will respond to the following
questions:

\begin{itemize}
  \item Did you find the tool's suggestions helpful? If not, why?
  \item Was the ordering of the tool's suggestions valid? If not, why?
\end{itemize}

\section{Related Work}

Roy et al provide a survey of Code Clone detection
techniques~\cite{Roy2009}.

Kawaguchi et al developed a Microsoft Visual Studio interface for
displaying code clones in real-time to support software
\emph{maintenance} tasks~\cite{Kawaguchi2009,Yamashina2008}. The
SHINOBI system uses the CCFinderX's preprocessor and the Suffix Array
technique for indexing clones. Displayed clones are ranked via the sum
of the ratio files committed at the same time and the ratio of files
opened or edited at the same period in Visual Studio. \todo{How do
  they get the latter piece of information?}

The work is motivated by a sliding window analysis in
\cite{Yamashina2008} of a commercial CAD application finding that
79.3\% of commits included modifications to files containing code
clones, but that only 9.7\% of such commits included modifications to
files containing the other clones. Additionally in
\cite{Yamashina2008}, Yamashina et al make some tenuous claims based
on interviews and observation that novice developers typically fix a
defect first, and then have difficulty finding and revising clones,
while experienced developers attempt to find all clones first (often
with difficulty when identifiers have changed), and then make changes.

\paragraph{Real-time Code Clone Search}

Keivanloo et all describe SeClone, a system for Internet code clone
search that performs clone pair clustering based on a ontology base on
features such as similarity~\cite{Keivanloo2011}. Similar to CCFinder,
it preprocesses files by generating the AST and abstracting the
tokens. The code patterns are used to quickly perform search, false
positives are limited by a retained set of type information. Results
are clustered via file-level type information.




\section{Conclusion}

Some concluding remarks

%
% The following two commands are all you need in the
% initial runs of your .tex file to
% produce the bibliography for the citations in your paper.
\bibliographystyle{abbrv}
\bibliography{rt-refactoring-proposal,bibstring-abbrev,ernst,invariants,types}  % sigproc.bib is the name of the Bibliography in this case
% You must have a proper ".bib" file
%  and remember to run:
% latex bibtex latex latex
% to resolve all references
%
% ACM needs 'a single self-contained file'!
%
\end{document}
