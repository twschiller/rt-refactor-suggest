\documentclass[preprint,10pt]{sigplanconf}

\newcommand{\todo}[1]{{\bfseries [[#1]]}}
%% To disable, just uncomment this line
%% \renewcommand{\todo}[1]{\relax}

\begin{document}
%
% --- Author Metadata here ---
\conferenceinfo{CSE503}{'11 Seattle, USA}
%\CopyrightYear{2007} % Allows default copyright year (20XX) to be over-ridden - IF NEED BE.
%\crdata{0-12345-67-8/90/01}  % Allows default copyright data (0-89791-88-6/97/05) to be over-ridden - IF NEED BE.
% --- End of Author Metadata ---

\title{Real-time Code Clone Refactoring Recommendations}
% 1st. author
\authorinfo{Travis Mandel}
           {University of Washington}
           {tmandel@cs.washington.edu}
% 2nd. author
\authorinfo{Todd W. Schiller}
           {University of Washington}
           {tws@cs.washington.edu}
\maketitle
\begin{abstract}
In the past, code clones detection and analysis has been viewed as a
maintenance problem. In this paper, we propose a tool and evaluation
for eliminating the introduction of code clones during development,
and leverage code similarity to boost programmer productivity. The tool
is an Eclipse plugin providing real-time clone detection and action
suggestions to the developer as (s)he writes and modifies
code. Machine learning is employed to provide more pertinent results
and suggestions to the developer.
\end{abstract}

\category{D.2.6}{Software Engineering}{Programming Environments}

\keywords{refactoring, recommender system, code clones}

\section{Introduction}

Numerous studies suggest that code clones impair the maintainability
of software.

Yamashina et al. found in a sliding window analysis of a commercial CAD
application that 79.3\% of commits included modifications to files
containing code clones, but that only 9.7\% of such commits included
modifications to files containing the other
clones suggesting some clones may have erroneously not been updated
~\cite{Yamashina2008}. 
%Additionally, they report tenuous
%evidence from interviews and observation both novice and experience
%developers have difficuly finding code clones (the latter when
%identifiers have changed), and that novice developers do not
%systematically find all code clones before beginning to make
%revisions.

Under the assumption that code clones are not maintained properly,
Jeurgens et al. built a static bug detection tool based on
inconsistencies in clones, and confirmed that clones were a major
source of bugs in the studies subject programs~\cite{Juergens2009}.
It seems reasonable to believe that if code maintenance of clones 
introduces bugs, clones are also a common source of bugs when they are
first written, because the older clone has undergone significantly more 
testing than the newer clone.

Code clones are typically viewed as a problem of software
\emph{maintenance}, as failure to revise a clone can be an error. In
addition to helping developers maintain clones, this work aims to
identify (1) potential method calls and opportunities for method
extraction, eliminating the introduction of unnecessary clones, and
(2) opportunities for copy-paste coding. Both reduce development cost 
due to rewriting and debugging duplicated code,
while the first also reduces maintenance cost.  The first also
improves code consistency and conventions by unifying
code written by different developers, and often improves code readability and 
modularity.

\section{Finding Clones}

% Don't use ``we'' to refer to the tool. The tool is the tool.

As the programmer develops, the tool will analyze the code to
determine the location of code clones, to aid in refactoring (method
extraction), method calls, or copying. In the future, the tool could
be extended to other refactorings / uses.

More concretely, the tool will search through the previous codebase for code that looks like a good match to the region that is currently being working on (i.e. where the cursor is). This could be implemented in a variety of ways (see section~\ref{sec:related}), from simple text-based matching to syntax tree based techniques. The tool won't require the code to compile, so some of the more advanced techniques that require a full parse of the source will not be used. In order to be practical in an online setting with a large codebase, we require that the method be fast and robust to identifying clones that are more obfuscated than simple copy-and-paste, such as when a programmer re-implements the same functionality without referring to the first code section.

\subsection{Determining When to Make Suggestions}
Once the tool has identified the clones, it will score each one based on how difficult they would be to extract, and how worthwhile the extraction would be. This score will be based on a variety of heuristics:


\begin{itemize}
  \item length of the code clone
  \item number of parameters in the extracted method
  \item module distance (how far away the clone is in the code base)
  \item how complete the current code section is
\end{itemize}

If the score passes a threshold, the tool will present the refactoring suggestions to the user.   In many cases it may suffice simply to show the "top hit", but in others it may be helpful to show a larger ranking.  The exact manner in which it presents the suggestion(s) to the programmer is still undecided, but we want to make sure it is relatively intrusive so that programmers don't just ignore the information. Popping up a dialog box is an option, as is highlighting the relevant block of code. We also plan to investigate, in place of a fixed threshold, simple machine learning strategies to ensure the false positives are reduced if the user frequently rejects the tool's suggestions.  This will take the form of a classifier using the heuristics as input, and may be trained on each individual user to account for different programming styles.

\section{Evaluation}

%We plan to evaluate based on user studies to determine how helpful our
%suggestions are.  Each user will be presented with an unfamiliar
%codebase and asked to implement a new method involving several We will
%record how many false positives there are, how many accepted
%suggestions there are, and how many times the user uses a method we
%extracted. We will record amount of code typed and amount of time
%spent. We will poll users after the fact to ask them how helpful they
%found the tool.

We will evaluate the tool via user case study. Each user will be given
a series of development and maintenance tasks for an existing Java
code base. The control group will use the stock Eclipse development
environment, the tool group will use the Eclipse development
environment with the tool. For the initial study for CSE 503, each
group will consist of two individuals who have experience using the
Java development environment.

Users will be instructed that they ``own'' the code base, that is they
have permission to introduce new methods, but must document the
methods.  They will be given a suite of unit tests convering the entire 
codebase (and their new development task) which must all pass
before the task is considered complete.

\subsection{Quantitative Evaluation}
For the development tasks, we will measure the time to complete the
task, the number of methods extracted, and the number of code clones
introduced into the code base, and percentage of the new code that is
a clone of other code. For the control group, we will additionally
record the number of times the developer searches for a method that
performs a subtask. For the tool group, we will record the number of
times the tool detects a code clone, the user ignores the tool's
alert, the user views the tool's analysis, and the user acts on the
tool's analysis. Additionally, when the user acts on the tool's
analysis, we will record the ranking of the clone the user acted on.

For the maintenance tasks, we will measure the time to complete the
task, and whether the user has correctly revised all of the clones (as
determined by hidden test suite). We will record the same
group-specific metrics as for the development tasks.

\subsection{Qualitative Evaluation}
In addition to the quantitative results, we will have the study
participants respond to the following questions:

\begin{itemize}
  \item If you extracted a method, how did you decide to extract the
    method? 
  \item If you chose not to extract a method, why did you decide not
    to extract the method?
\end{itemize}

\noindent Additionally, the treatment group will respond to the following
questions:

\begin{itemize}
  \item Did you find the tool's suggestions helpful? If not, why?
  \item Was the ordering of the tool's suggestions valid? If not, why?
\end{itemize}

\subsection{Threats to Validity}
The evaluation outline in this section is meant as a preliminary study
to investigate the efficacy of the tool --- it is not meant to be
conclusive. That being said, this study, and potentially larger
studies of the same design have the following potential threats to
validity:

\begin{itemize}
  \item The study participants do not have to maintain the code in the
    future, and are therefore may be more likely to perform a
    short-sighted action
  \item The results may not generalize, as in real software there may
    be de facto or de jure restrictions (e.g., a certain module cannot
    be changed) restricting the set of actions a user can make
  \item  This tool seems especially helpful if it suggests that a person is 
    cloning code which that person personally wrote in the past.  Due to time 
    limitations this scenario may be difficult to induce, as it usually occurs after
    one has been working on the same project for a long time.
\end{itemize}

We believe that the first threat can be mitigated via instructions to
the study participants. The second factor may require an additional
investigation where such restrictions are in place.  The third factor
requires a long-term study over weeks, months, or years.

\section{Related Work}
\label{sec:related}

The vast majority of the code clone literature can be divided into two areas: (1)
\emph{how} to detect code clones, and (2) how to best visualize the
results of post-mortem tools, which operate over an entire solution.
The latter varies based upon usage, e.g., for detecting plagiarism
versus development usage.

Roy et al provide a survey of code clone detection
techniques~\cite{Roy2009}. The rest of this section surveys the
related work on recommender systems, code clone interfaces, and
\emph{real-time} code clone detection.

\paragraph{Recommender Systems}

Holmes and Murphy built the Strathcona tool for Eclipse which displays
relevant API usage examples to a user based on the structural context
of a query line(s) of code the user selects in the IDE
\cite{Holmes2005}. Related systems also exist, but require the user to
perform a formal query, or to write special comments in the code.

\paragraph{Code Clone Interfaces}

Visual Studio Ultimate contains a code clone detection tool and
interface; the tool can be run on a particular code fragment, or over
the entire solution~\cite{VSClones}. Other commercial products and
academic artifacts exist which provide different interfaces /
visualizations.

Perhaps the work most similar to ours is Kawaguchis et al.'s 
Microsoft Visual Studio interface for
displaying code clones in real-time to support software
\emph{maintenance} tasks~\cite{Kawaguchi2009,Yamashina2008}. Their
SHINOBI system uses the CCFinderX's preprocessor and the Suffix Array
technique for indexing clones. Displayed clones are ranked via the sum
of the ratio files committed at the same time and the ratio of files
opened or edited at the same period in Visual Studio. \todo{How do
  they get the latter piece of information?}

\paragraph{Real-time Code Clone Search}

Keivanloo et al. describe SeClone, a system for Internet code clone
search that performs clone pair clustering based on a ontology base on
features such as similarity~\cite{Keivanloo2011}. Similar to CCFinder,
it preprocesses files by generating the AST and abstracting the
tokens. The code patterns are used to quickly perform search, false
positives are limited by a retained set of type information. Results
are clustered via file-level type information.

Lee et al. introduce a method for instant structural code clone search
over large repositories by utilizing an R*tree indexing structure over
the characteristic vectors~\cite{Lee2010}.

\section{Conclusion}

In the past, code clones detection and analysis has been viewed as a
maintenance problem. We believe the problems code clones pose are 
more serious, especially those that are re-written instead of being a 
"copy and paste".  We have proposed a tool and evaluation for
eliminating the introduction of code clones during development, and
levering similar code to boost programmer productivity and improve
code robustness and organization.

%
% The following two commands are all you need in the
% initial runs of your .tex file to
% produce the bibliography for the citations in your paper.
\bibliographystyle{abbrv}
\bibliography{rt-refactoring-proposal,bibstring-abbrev,ernst,invariants,types}  % sigproc.bib is the name of the Bibliography in this case
% You must have a proper ".bib" file
%  and remember to run:
% latex bibtex latex latex
% to resolve all references
%
% ACM needs 'a single self-contained file'!
%
\end{document}
