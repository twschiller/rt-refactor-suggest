\title{Quick Photo Instructions}

\documentclass[12pt]{article}
\usepackage{fullpage}

\newcommand{\todo}[1]{{\bfseries [[#1]]}}
%% To disable, just uncomment this line
%\renewcommand{\todo}[1]{\relax}

\begin{document}
\maketitle

\section{Introduction}
Today you will be playing the role of one of the developers of
QuickPhoto, a small application for manipulating images. QuickPhoto is 
designed to deal with higher quality images than those currently used:
Instead of having only 256 values for the red channel, QuickPhoto is designed
to run on images with thosands or millions of colors of pure red. 

Unfortunately, a standard format for representing colors in this space has not
yet been agreed upon.  As such, all code must be flexible enough to deal with
colors at different scales.  The code may only make the following assumptions:

\begin{enumerate}
\item Each color channel is represented as an int.
\item Colors are rougly centered on 0: That is: 0,0,0 is guaranteed to be gray.
\item The scale of colors is constant across an image and across channels:
	(23, -40, 400) is always the same blue-ish color no matter where it appears.
\end{enumerate}

Your
development team has just installed a new Eclipse plugin for working
with duplicate code, and is vetting the tool for use during software
development and maintenance.

Over the next 40 - 60 minutes, you will be adding a new feature to the
QuickPhoto application, and fixing a bug that a user has found. After
performing these two tasks, you will answer questions about the tasks
you just performed.

\section{Application Overview}
The codebase currently has a GUI frontend which allows you to apply a variety of 
transformations to a square image of your choosing.  This GUI is implemented in the classes
MainImage and PaintShop, and you will not need to concern yourself with the details of their 
implementation.

QuickPicture and QuickColor are the classes used to present our high-fidelity images and colors.
QuickColors have an integer red, green, blue and alpha channel.

The transforms package contains a variety of complex image transformations which we 
have implemented thus far.  Each transformation takes as input a QuickPicture and outputs 
the transformed QuickPicture.  Any additional information needed for the transformation 
is  passed in via the constructor.

ImageUtil contains several simple transformations which could be useful as a step in other 
transformations, namely flip and shrink.

You should feel free to modify any source files EXCEPT the classes in the ``eval" package
and the test cases.

\section{Duplicate Code Plugin}
The duplicate code plugin detects when similar code appears in
multiple places in a project. For each region of similar code, it may
recommend four different actions:

\begin{itemize}
  \item Insert a method call to the method containing the similar code
  \item Create a new method that contains the similar code (method extraction)
  \item Open an editor that is focused on the similar code, and
  \item Paste the similar code, substituting identifiers as needed
\end{itemize}

\paragraph{Plugin Modes}
The tool has a development mode, and a maintenance mode; these modes
can be toggled using the toolbar. In development mode, the tool only
marks similar code in the region where the last file edit happened
(where you are developing). In maintenance mode, the tool marks all of
the similar code in the active file.

\section{Development (approx. 20 minutes)}

You are tasked with completing the implementation of the ``NewImageTransform" transformation.  This transformation has two stages.
In the first stage, you must go through each pixel and assign its color to be the average of each of its four diagonal neighbors.
In the second stage, you must write out the image id indicated by the field ``imageId" to the upper left corner of the image.
To simplify the task, the number needs only to be written out in binary.  Here is the precise definition of how to write out the two numerals, 1 and 0:

\begin{itemize}
\item The upper left corner of the numerals region starts at (5, height-30), so the number is printed in the lower left of the image.
\item The characters all get a 15 pixel space to work with before the next character begins.
\item The zero is a rectangle 20 pixels tall and 10 pixels  wide, flush to the left side of its space.
\item The one is a single vertical line, indented 5 pixels into its space (i.e. pixel index 5) and 20 pixels tall.
\item The RGB color of the lines must be equal to the ``idColor" field, but the alpha channel MUST be preserved.
\end{itemize}

Again, for this task, all alpha values must not be modified during the transformation. 
They must be the same as the original image.


You have been tasked with adding a new feature to QuickPhoto.


\section{Maintenance (approx. 20 minutes)}

A user has found a bug with the static method ``ImageUtil.downsizeImage".  When an image with a variety of alpha values 
is passed in, the alpha values of the resulting image do not look quite right.  downsizeImage, and any other code code which averages over a block of values,  is supposed to average the alphas as well as the RGB values.  Please fix this bug.  If you 
happen to notice the same bug elsewhere in the codebase, feel free to modify it there as well.


\section{Reflection (approx. 20 minutes)}

Please type the answers the following questions about the
recommendations provided by the tool, and the actions that you
performed when performing the development and maintenance tasks.

\paragraph{Similar Code Suggestions}

\begin{enumerate}
  \item Did you find the duplicate code tool's suggestions helpful?
    Why, or why not? Does your answer differ for the development and
    maintenance tasks? If so, why?
  \item Was the ranking (ordering) of the tool's suggestions valid?
    Why, or why not? Does your answer differ for the development and
    maintenance tasks? If so, why?
\end{enumerate}

\paragraph{Method Extraction}

\begin{enumerate}
  \item If you extracted a method, how did you decide to extract the
    method? 
  \item If you chose not to extract a method, why did you decide not
    to extract the method?
\end{enumerate}

\end{document}
