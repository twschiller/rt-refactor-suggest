\title{QuickPhoto Instructions}

\documentclass[12pt]{article}
\usepackage{fullpage}

\newcommand{\todo}[1]{{\bfseries [[#1]]}}
%% To disable, just uncomment this line
%\renewcommand{\todo}[1]{\relax}

\begin{document}
\maketitle

\section{Introduction}
Today you will be playing the role of one of the developers of
QuickPhoto, a small application for manipulating square images. Unlike
traditional image tools, QuickPhoto works with high-fidelity images
which are not limited to 256 colors per channel (e.g., there are many
thousand variants of the color ``red'').

Your
development team has just installed a new Eclipse plugin for working
with duplicate code, and is vetting the tool for use during software
development and maintenance. 
Over the next 40 - 60 minutes, you will be adding a new feature to the
QuickPhoto application, and fixing a bug that a user has found. After
performing these two tasks, you will answer questions about the tasks
you just performed.

\section{Application Overview}
QuickPhoto is a GUI application which allows users to apply a variety
of transformations to a square image of their choosing. The GUI is
implemented in \verb|MainImage| and \verb|PaintShop|; you do not need
to concern yourself with the details of their implementation. To run
the application, run the \verb|MainImage| class. Take a few minutes to
run the application, trying a few image transformations on images in the 
\verb|img| folder.

\paragraph{Colors}
\verb|QuickPicture| and \verb|QuickColor| are the classes used to
present high-fidelity images and colors.  Like traditional colors,
\verb|QuickColor|s have an integer red, green, blue and alpha
(transparency) channel.

Unfortunately, a standard format for representing colors in this space
has not yet been agreed upon.  As such, all code must be flexible
enough to deal with colors at different scales; The code may only make
the following assumptions:

\begin{enumerate}
\item Each color channel is represented as an integer (\verb|int|).
\item Colors are roughly centered at 0, i.e. \verb|(0, 0, 0)| is guaranteed
  to be gray,
\item The scale of colors is consistent across an image and across
  channels, e.g., \verb|(23, -40, 400)| is always encodes same blue-ish color
\end{enumerate}

\paragraph{Image Transformations}

The \verb|transform| package contains a variety of complex image
transformations.  Each transformation implements the
\verb|ImageTransform| interface, which provides a method that takes as
input a \verb|QuickPicture| and outputs the transformed
\verb|QuickPicture|. Any additional information needed for the
transformation is passed in via the transformation's constructor.

The class \verb|ImageUtil| contains several simple transformations
which could be useful as a step in other transformations, namely flip
and shrink.

\paragraph{Transformation Tests}

The \verb|tests| package contains JUnit tests for the
transformations. Each test consists of at least one original image and
target image. 
All tests pass initially except \verb|testNewTransform|
and \verb|testShrink|, which should pass after you have completed
the development and maintenance tasks, respectively.
If you have questions about how to run the unit tests,
please ask the proctor before beginning the tasks.


\paragraph{Code Ownership}  
You should feel free to modify any source files \textit{EXCEPT} the classes in
the \verb|eval| package and the test cases.  If you choose to add any methods,
you must document the methods.

\section{Duplicate Code Plugin}
The duplicate code plugin detects when similar code appears in
multiple places in a project. For each region of similar code, it may
recommend four different actions:

\begin{itemize}
  \item Insert a method call to the method containing the similar code
  \item Create a new method that contains the similar code (method extraction)
  \item Open an editor that is focused on the similar code, and
  \item Paste the similar code, substituting identifiers as needed
\end{itemize}

\paragraph{Plugin Modes}
The tool has a development mode, and a maintenance mode; these modes
can be toggled using the toolbar button. In development mode, the tool only
marks similar code in the region where the last file edit happened
(where you are developing). In maintenance mode, the tool marks all of
the similar code in the active file.

\pagebreak

\section{Development (approx. 20 minutes)}

You are tasked with completing the implementation of the
\verb|NewImageTransform| transformation.  This transformation should
work in two stages, as described below.

\paragraph{Stage 1}
In the first stage, the transformation iterates over each
pixel and assign its color to be the average (mean) of each of its
four diagonal neighbors. If a pixel is missing a diagonal neighbor,
assume the neighbor exists with RGB values \verb|(0, 0, 0)|.

\paragraph{Stage2}
In the second stage, the transformation
writes the image id indicated by the field \verb|imageId| \textit{in binary} to the lower
left corner of the image.  To convert the number to a sequence of binary digits, use the 
\verb|Integer.toBinaryString()| method. The numerals \verb|0|
and \verb|1| should be written according to the following specification:

\begin{itemize}
\item The binary string is printed in the lower-left of the image.
  The upper-left corner of the numerals starts at column \verb|5|, row
  \verb|height - 30|.
\item Each character is written in a bounding box that is 20 pixels
  tall and 15 pixels wide.
\item The zero (\verb|0|) is drawn as a rectangle 20 pixels tall and
  10 pixels wide, flush with the left side of its space.
\item The one (\verb|1|) is a single vertical line, indented 5 pixels
  into its bounding box (i.e. pixel offset 5) and 20 pixels tall.
\item The RGB color of the lines must be equal to the \verb|idColor|
  field, but the alpha channel MUST be preserved.
\end{itemize}

\paragraph{Preserving the Alpha Channel}
Again, the transformation must not modify the alpha values -- i.e.,
the alpha channels in the transformed image must be identical to the
alpha channels in the original image.

\paragraph{Testing}
When you have completed this task successfully,
\verb|testNewTransform| will pass.

\pagebreak

\section{Maintenance (approx. 20 minutes)}

A user has found a bug with the static method
\verb|ImageUtil.downsizeImage|.  When an image with a variety of alpha
values is passed in (such as  \verb|Ghost.png|), the alpha values of the resulting image do not
look quite right.  \verb|ImageUtil.downsizeImage|, \textit{and any
  other code code which averages over a block of values}, is supposed
to average (mean) the alphas as well as the RGB values.  Please fix this bug
in \verb|ImageUtil.downsizeImage|; If you happen to notice the same
bug elsewhere, feel free to correct the code there as well.

\paragraph{Testing} When you have fixed the bug successfully, \verb|testShrink| will pass.

\pagebreak

\section{Reflection (approx. 20 minutes)}

Please type the answers the following questions about the
recommendations provided by the tool, and the actions that you
performed when performing the development and maintenance tasks.

\paragraph{Similar Code Suggestions}

\begin{enumerate}
  \item Did you find the duplicate code tool's suggestions helpful?
    Why, or why not? Does your answer differ for the development and
    maintenance tasks? If so, why?
  \item Was the ranking (ordering) of the tool's suggestions valid?
    Why, or why not? Does your answer differ for the development and
    maintenance tasks? If so, why?
  \item In which situation(s), if any, would this tool be most useful
  to you as a programmer?
\end{enumerate}

\paragraph{Method Extraction}

\begin{enumerate}
  \item If you extracted a method, how did you decide to extract the
    method? 
  \item If you chose not to extract a method, why did you decide not
    to extract the method?
\end{enumerate}

\end{document}
